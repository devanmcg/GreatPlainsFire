% Options for packages loaded elsewhere
\PassOptionsToPackage{unicode}{hyperref}
\PassOptionsToPackage{hyphens}{url}
%
\documentclass[
]{article}
\usepackage{amsmath,amssymb}
\usepackage{iftex}
\ifPDFTeX
  \usepackage[T1]{fontenc}
  \usepackage[utf8]{inputenc}
  \usepackage{textcomp} % provide euro and other symbols
\else % if luatex or xetex
  \usepackage{unicode-math} % this also loads fontspec
  \defaultfontfeatures{Scale=MatchLowercase}
  \defaultfontfeatures[\rmfamily]{Ligatures=TeX,Scale=1}
\fi
\usepackage{lmodern}
\ifPDFTeX\else
  % xetex/luatex font selection
\fi
% Use upquote if available, for straight quotes in verbatim environments
\IfFileExists{upquote.sty}{\usepackage{upquote}}{}
\IfFileExists{microtype.sty}{% use microtype if available
  \usepackage[]{microtype}
  \UseMicrotypeSet[protrusion]{basicmath} % disable protrusion for tt fonts
}{}
\makeatletter
\@ifundefined{KOMAClassName}{% if non-KOMA class
  \IfFileExists{parskip.sty}{%
    \usepackage{parskip}
  }{% else
    \setlength{\parindent}{0pt}
    \setlength{\parskip}{6pt plus 2pt minus 1pt}}
}{% if KOMA class
  \KOMAoptions{parskip=half}}
\makeatother
\usepackage{xcolor}
\usepackage[margin=1in]{geometry}
\usepackage{graphicx}
\makeatletter
\def\maxwidth{\ifdim\Gin@nat@width>\linewidth\linewidth\else\Gin@nat@width\fi}
\def\maxheight{\ifdim\Gin@nat@height>\textheight\textheight\else\Gin@nat@height\fi}
\makeatother
% Scale images if necessary, so that they will not overflow the page
% margins by default, and it is still possible to overwrite the defaults
% using explicit options in \includegraphics[width, height, ...]{}
\setkeys{Gin}{width=\maxwidth,height=\maxheight,keepaspectratio}
% Set default figure placement to htbp
\makeatletter
\def\fps@figure{htbp}
\makeatother
\setlength{\emergencystretch}{3em} % prevent overfull lines
\providecommand{\tightlist}{%
  \setlength{\itemsep}{0pt}\setlength{\parskip}{0pt}}
\setcounter{secnumdepth}{-\maxdimen} % remove section numbering
\usepackage{amsmath}
\renewcommand{\familydefault}{\sfdefault}
\usepackage[colorlinks=true, citecolor=blue, urlcolor=blue,linkcolor=blue, pdfborder={0 0 0}]{hyperref}
\urlstyle{same}
\usepackage{xcolor}
\usepackage{booktabs}
\raggedright
\usepackage{listings}
\ifLuaTeX
  \usepackage{selnolig}  % disable illegal ligatures
\fi
\usepackage{bookmark}
\IfFileExists{xurl.sty}{\usepackage{xurl}}{} % add URL line breaks if available
\urlstyle{same}
\hypersetup{
  pdftitle={Evaluating an attempt to restore summer fire in the Northern Great Plains},
  pdfauthor={Devan Allen McGranahan and Jay P. Angerer},
  hidelinks,
  pdfcreator={LaTeX via pandoc}}

\title{Evaluating an attempt to restore summer fire in the Northern
Great Plains}
\usepackage{etoolbox}
\makeatletter
\providecommand{\subtitle}[1]{% add subtitle to \maketitle
  \apptocmd{\@title}{\par {\large #1 \par}}{}{}
}
\makeatother
\subtitle{Supplementary Information}
\author{Devan Allen McGranahan and Jay P. Angerer}
\date{}

\begin{document}
\maketitle

This document provides script for fetching remotely-sensed imagery and
processing results in the \textsf{R} statistical environment.

\section{Remote sensing}\label{remote-sensing}

\subsection{Sentinel-2}\label{sentinel-2}

This \texttt{EvalScript} can be used in the
\href{https://browser.dataspace.copernicus.eu/}{Copernicus browser} as a
Custom script to create a Custom visualization that can be exported as a
16-bit \texttt{.tiff} for raster analysis. The user must select the area
of interest and imagery dates and is responsible for assessing
cloudiness.

\begin{lstlisting}[language=Java]

//VERSION=3
function setup() {
  return {
    input: ["B04", "B08", "B8A", "B11", "B12"],
    output: { bands: 2, sampleType: "UINT16" }
  };
}

function evaluatePixel(sample) {
  let nbr = index(sample.B08, sample.B12);
  let ndvi = index(sample.B08, sample.B04);
  // apply offset for UINT16 
  return [10000 * nbr + 10000, 
          10000 * ndvi + 10000]; 
}
\end{lstlisting}

\clearpage

\subsection{Landsat}\label{landsat}

This script for \href{https://code.earthengine.google.com/}{Google Earth
Engine} exports a composited NDVI \texttt{.tiff} file to the user's
Google Drive for the area of interest defined as \texttt{geometry} for
an eight week period beginning with each date given in
\texttt{listDate}. The script combines imagery from Landsat missions 5,
7, \& 8 and handles cloud masking.

\begin{lstlisting}[language=Java]
// Geometry
var geometry: Polygon, 4 vertices
  type: Polygon
  coordinates: List (1 element)
    0: List (5 elements)
      0: [-99.51276195869765,46.71741699210676]
      1: [-99.42521465645156,46.71741699210676]
      2: [-99.42521465645156,46.778346255312734]
      3: [-99.51276195869765,46.778346255312734]
      4: [-99.51276195869765,46.71741699210676]
  geodesic: false

// Main script

var batch = require('users/fitoprincipe/geetools:batch');

///****variables that need to be changed by user
//****ADD google drive directory name that you want to download files to
var gdrivedir = 'landsat'

var listDate= ["1992-04-20", "1992-07-20", "1993-04-20", "1993-07-20", 
               "1994-04-20", "1994-07-20", "1995-04-20", "1995-07-20", 
               "1996-04-20", "1996-07-20", "1997-04-20", "1997-07-20", 
               "1998-04-20", "1998-07-20", "1999-04-20", "1999-07-20", 
               "2000-04-20", "2000-07-20", "2001-04-20", "2001-07-20", 
               "2002-04-20", "2002-07-20", "2003-04-20", "2003-07-20",
               "2004-04-20", "2004-07-20", "2005-04-20", "2005-07-20", 
               "2006-04-20", "2006-07-20", "2007-04-20", "2007-07-20", 
               "2008-04-20", "2008-07-20", "2009-04-20", "2009-07-20",
               "2010-04-20", "2010-07-20", "2011-04-20", "2011-07-20", 
               "2012-04-20", "2012-07-20", "2013-04-20", "2013-07-20", 
               "2014-04-20", "2014-07-20", "2015-04-20", "2015-07-20",
               "2016-04-20", "2016-07-20", "2017-04-20", "2017-07-20", 
               "2018-04-20", "2018-07-20", "2019-04-20", "2019-07-20", 
               "2020-04-20", "2020-07-20", "2021-04-20", "2021-07-20",
               "2022-04-20", "2022-07-20"]

listDate.forEach(function (listDate) {
  //yearRanges.forEach(function (yearRange) {
    exportTimeseries(listDate)
  })
//})
function exportTimeseries(listDate) {
  
  // Defines a base date/time for the following examples.
  var startDate = ee.Date(listDate);
  print(startDate, 'The start date/time');
  print(startDate.format('YYYYMMdd'))
  var sdate = startDate.format('YYYYMMdd')
  var endDate = startDate.advance(8, 'week')
  print(endDate, 'The end date/time');
  var edate = endDate.format('YYYYMMdd')
  
  print(endDate.format('YYYYMMdd'))

  //from https://developers.google.com/earth-engine/tutorials/
  //                community/extract-raster-values-for-points
  //function to mask cloud and shadow pixels
  function fmask(img) {
    var cloudShadowBitMask = 1 << 4;
    var cloudsBitMask = 1 << 3;
    var qa = img.select('QA_PIXEL');
    var mask = qa.bitwiseAnd(cloudShadowBitMask).eq(0)
      .and(qa.bitwiseAnd(cloudsBitMask).eq(0));
    return img.updateMask(mask);
  }
  
  // Selects and renames bands of interest for Landsat OLI.
  function renameOli(img) {
    return img.select(
      ['SR_B2', 'SR_B3', 'SR_B4', 'SR_B5', 'SR_B6', 'SR_B7'],
      ['Blue', 'Green', 'Red', 'NIR', 'SWIR1', 'SWIR2']);
  }
  
  // Selects and renames bands of interest for TM/ETM+.
  function renameEtm(img) {
    return img.select(
      ['SR_B1', 'SR_B2', 'SR_B3', 'SR_B4', 'SR_B5', 'SR_B7'],
      ['Blue', 'Green', 'Red', 'NIR', 'SWIR1', 'SWIR2']);
  }
  
  // Prepares (cloud masks and renames) OLI images.
  function prepOli(img) {
    img = fmask(img);
    //img = lcfmask(img)
    img = renameOli(img);
    return img;
  }
  
  // Prepares (cloud masks and renames) TM/ETM+ images.
  function prepEtm(img) {
    img = fmask(img);
    //img = lcfmask(img)
    img = renameEtm(img);
    return img;
  }
  
  // Apply scaling factors
  function applyScaleFactors(image) {
    var opticalBands = image.select('SR_B.').multiply(0.0000275).add(-0.2);
    var thermalBands = image.select('ST_B.*').multiply(0.00341802).add(149.0);
    return image.addBands(opticalBands, null, true)
                .addBands(thermalBands, null, true);
  }
  
  // Get surface reflectance collections for Landsat
  // maps scaling factors and cloud masking functions 
  
  var oliCol = ee.ImageCollection('LANDSAT/LC08/C02/T1_L2')
              .filter(ee.Filter.bounds(geometry))
              .map(applyScaleFactors)
              .map(prepOli);
  
  var etmCol = ee.ImageCollection('LANDSAT/LE07/C02/T1_L2')
                .filter(ee.Filter.bounds(geometry))
                .map(applyScaleFactors)
                .map(prepEtm);
  
  var tmCol = ee.ImageCollection('LANDSAT/LT05/C02/T1_L2')
              .filter(ee.Filter.bounds(geometry))
              .map(applyScaleFactors)
              .map(prepEtm) ;
  
  var landsatCol = oliCol.merge(etmCol).merge(tmCol)
                   .filterDate(startDate, endDate) ;
  
  // Calculate NDVI 
  var addIndices = function(image) {
     var ndvi = image.normalizedDifference(['NIR', 'Red'])
                .rename('ndvi').float();
  return image.addBands([ndvi]);
  };
  
  var indices = landsatCol.map(addIndices) 
                .select('ndvi');
  // Create composite of images from date range
    var composite = indices.mean() ; 
  
    var fileName = ee.Date(startDate) 
                  .format('yyyy-MM-dd')
                  .getInfo() ;

  Export.image.toDrive({
      image: composite,
      description: fileName,
      scale: 30,
      folder: 'landsat', 
      region: geometry
      });
}
\end{lstlisting}

\newpage

\section{Analysis}\label{analysis}

Because the analysis script assumes hundreds of imagery files and a
large Excel file with 42 years of hourly weather observations have been
saved locally, it cannot be run directly from this document but is
provided here for transparency and reference.

\subsection{Remotely sensed imagery} 

The following \textsf{R} script processes the imagery produced by the scripts above after they have been saved to a drive that can be mapped to from the session as \texttt{imagery\_dir}. 

\begin{lstlisting}[language=R] 

# Load necessary packages
# Note that package terra is required but is not loaded
# (called directly to not create conflicts with dplyr verbs)
  pacman::p_load(tidyverse, sf, stars, foreach, doSNOW)

# Load location boundaries (data not provided)
  cgrec_gpkg = './CGREC_PBG_26914.gpkg'

  pastures <- st_read(cgrec_gpkg, 'Pastures') 
  patches <- st_read(cgrec_gpkg, 'PasturePatches') 
#
# Get veg data for unburned pastures
#
  NoFirePts <- st_read(cgrec_gpkg, 'SamplePoints') %>%
                filter(location == 'Refuge')
  # Map to directory with Sentinel-2 imagery
    imagery_dir = 'C:/Path/To/Sentinel'
    images <- list.files(imagery_dir)
  # Use parallel processing to chug imagery
  { 
    begin = Sys.time() 
    pacman::p_load()
    cores = parallel::detectCores()
    cl <- makeCluster(cores, methods = F, useXDR = F)
    registerDoSNOW(cl)
    NoFireIndices <-
      foreach(i=1:length(images), 
              .combine = 'bind_rows',
              .errorhandling = 'remove', 
              .packages=c('tidyverse', 'sf')) %dopar% {
        image = images[i]
        image_path = paste0(imagery_dir, '/', image)
        ras <- terra::rast(image_path)
        names(ras) <- c('nbr', 'ndvi') 
        ras <- ras[['ndvi']]
        float = (ras-10000)/10000
        terra::extract(float,
          NoFirePts %>%
            select(pasture, sample) %>%
            terra::vect() , 
          FUN = mean, 
          bind = TRUE)  %>%
        st_as_sf() %>%
          as_tibble() %>%
        mutate(ImageDate = substr(image, 1, 10)) %>%
        select(ImageDate, pasture, sample, ndvi) 
            }
    stopCluster(cl)
  Sys.time() - begin 
  }
#
# Get fuel greenness and dNBR for completed burns
#
  fires <- st_read(cgrec_gpkg, 'FirePerimeters') %>%
              mutate(Year = as.factor(Year)) %>%
              filter(status == 'Completed') %>%
              unite('fire', c(unit, Pasture, Patch), sep = "-") %>%
              select(fire, Year, Season, PreBurn, PostBurn )
  SamplePts <- 
    st_read(cgrec_gpkg, 'SamplePointsRegular')
{ 
  begin = Sys.time() 
  pacman::p_load(foreach, doSNOW)
  cores = parallel::detectCores()
  cl <- makeCluster(cores , methods = F, useXDR = F)
  registerDoSNOW(cl)
  BurnIndices <-
    foreach(i=1:length(fires$fire), 
            .combine = 'bind_rows',
            .errorhandling = 'remove', 
            .packages=c('tidyverse', 'sf')) %dopar% {
              # Get fire
                fire = slice(fires, i)
              # Get sample points
                pts <-
                  fire %>%
                    st_intersection(SamplePts)
              # Get dates
                pre_date = fire$PreBurn
                post_date = fire$PostBurn
              # Fetch & process multi-band rasters
                # pre image
                  pre_image = images[substr(images, 1, 10) == pre_date]
                  pre_path = paste0(imagery_dir, '/', pre_image)
                  pre_ras <- terra::rast(pre_path) %>%
                                terra::crop(terra::vect(fire))
                  names(pre_ras) <- c('nbr', 'ndvi') 
                  pre_ras <-  (pre_ras-10000)/10000
                # post-fire
                  post_image = images[substr(images, 1, 10) == post_date]
                  post_path = paste0(imagery_dir, '/', post_image)
                  post_ras <- terra::rast(post_path)[[1]] %>%
                                terra::crop(terra::vect(fire))
                  post_ras = (post_ras-10000)/10000
                # Calculate dNBR & replace in pre-fire raster
                  d_ras = pre_ras['nbr'] - post_ras
                  names(d_ras) <- 'dNBR'
                  pre_ras[[1]] <- d_ras
              # Sample rasters
                 terra::extract( pre_ras,
                          pts %>%
                            select(-PostBurn) %>%
                            terra::vect() , 
                          FUN = mean, 
                          bind = TRUE)  %>%
                  st_as_sf() %>%
                    as_tibble() %>%
                    select(-geometry) 
            }
  stopCluster(cl)
  Sys.time() - begin 
}
#
# Get historical Landsat data for unburned pastures
#
  imagery_dir = 'C:/Path/To/Landsat'
  images <- list.files(imagery_dir)
  { 
    begin = Sys.time() 
    pacman::p_load(foreach, doSNOW)
    cores = parallel::detectCores()
    cl <- makeCluster(cores, methods = F, useXDR = F)
    registerDoSNOW(cl)
    NoFireTrends <-
      foreach(i=1:length(images), 
              .combine = 'bind_rows',
              .errorhandling = 'remove', 
              .packages=c('tidyverse', 'sf')) %dopar% {
                image = images[i]
                image_path = paste0(imagery_dir, '/', image)
                ras <- terra::rast(image_path)
                ras <- ras[['ndvi']]
                 terra::extract(ras,
                               NoFirePts %>%
                                 select(pasture, sample) %>%
                                 st_transform(4326) %>%
                                 terra::vect() , 
                               FUN = mean, 
                               bind = TRUE)  %>%
                  st_as_sf() %>%
                  as_tibble() %>%
                  mutate(ImageDate = tools::file_path_sans_ext(image))  %>%
                  select(ImageDate, pasture, sample, ndvi) 
              }
    stopCluster(cl)
    Sys.time() - begin 
  }
\end{lstlisting}

\newpage
\subsection{Weather data}\label{weather-data}

This script wrangles the weather data downloaded from the
\href{https://ndawn.ndsu.nodak.edu/station-info.html?station=48}{North
Dakota Ag Weather Network's Streeter station}.

\begin{lstlisting}[language=R] 

pacman::p_load(tidyverse, readxl)

seasons <- tibble(season = c('spring', 'summer'), 
                 start = c('04-20', '07-20'), 
                 end = c('06-10', '09-10') ) %>%
          mutate(start = as.Date(start, '%m-%d'), 
                 end = as.Date(end, '%m-%d'))

BurnDays <- 
  read_xlsx('./data/BurnData.xlsx', 'BurnDays') %>%
  filter(is.na(certainty)) %>%
    mutate(date = paste(year, date), 
           date = as.Date(date, '%Y %B %d')) %>%
    select(date) %>%
  distinct() 

WxData <- lst() 
# Get daily data 
  DailyWx <- 
    read_xlsx('./data/CGREC_weather.xlsx', 'daily')
# Identify rainy days
  WxData$Rainfall <-
    DailyWx %>%
    select(Year, Month, Day, Rainfall) %>% 
    unite( c(Month, Day), col = 'day', sep = '-', remove = F) %>%
    mutate(day = as.Date(day, '%m-%d')  , 
           season = case_when(
             between(day, seasons$start[1], seasons$end[1]) ~'Spring', 
             between(day, seasons$start[2], seasons$end[2]) ~'Summer',
             TRUE ~ NA      
           ) ) %>% 
     unite( c(Year, Month, Day), col = 'date', sep = '-')

# Hourly data 
  WxData$HourlyWx <- 
    read_xlsx('./data/CGREC_weather.xlsx', 'hourly') %>%
    filter(between(Hour, 1000, 1700)) %>%
    unite( c(Year, Month, Day), col = 'date', sep = '-', remove = F) %>%
    unite( c(Month, Day), col = 'day', sep = '-', remove = F) %>%
    mutate(day = as.Date(day, '%m-%d'), 
           season = case_when(
             between(day, seasons$start[1], seasons$end[1]) ~'Spring', 
             between(day, seasons$start[2], seasons$end[2]) ~'Summer',
             TRUE ~ NA      
           ) ) %>%
    mutate(e = 6.11 * (10 ^ ( (7.5 * DewPoint)/ (237.3 + DewPoint) ) ), 
           es = 6.11 * (10 ^ ( (7.5 * AirTemp)/ (237.3 + AirTemp) ) ), 
           VPD = es - e) 
# Get burn day weather
  WxData$BurnDayWx <-
    HourlyWx  %>% 
    filter(date %in% BurnDays$date) %>%
    group_by(Year, date, season) %>%
    summarise_at(.vars = vars(c(DewPoint, RelHum, WindSpeed,VPD)),
                 .funs = c("mean")) %>%
    ungroup() %>%
    pivot_longer(names_to = 'variable', 
                 values_to = 'value', 
                 cols = DewPoint:VPD) 
\end{lstlisting}

\end{document}
