\documentclass[parskip=half, 
			   fontsize=10pt,
			   paper=a4]				
{scrartcl}
\input{SetupOptions} % contains scrletter

\newcommand{\journal}{\emph{Environmental Management}}


\vspace{-5em} 

\begin{document}
	\begin{letter}{Editorial staff\\
		\journal }
\setlength{\parindent}{12pt}

\opening{Dear Sir or Madam,}  
		
Please consider the attached submission for publication in \journal. 
We are excited to leverage a unique dataset to address a novel question about the restoration of pre-European fire regimes: How successful are attempts at summer burning?
Much research attention has been given to variation in fire effects across different seasons, but to our knowledge no one has yet directly addressed the feasability of burning during the growing season, along wth a comparison of burn outcomes (not fire effects) relative to conditions of the fire environment. 

We also provide a historical context for the fuel and weather variables considered here, however the limited spatial resolution of the research station represents a narrow scope. 
Indeed, while the relatively high number of 40 fire events across four years provides plenty of food for thought, the present paper is best approached as a case study with a narrow scope in which we demonstrate how various data products can be used to assess patterns in fire activity and outcomes at broader scales.  

Please note that we have prepared and submitted this as a \textbf{Methodology} paper, for two basic reasons: Firstly, our emphasis is on the feasibility of conducting the treatments, not evaluating their ecological outcomes with reference to experimental hypotheses; and secondly, this is a case study focused on a single research station. 
Although 40+ fires is a decent sample size, we are reticent to extrapolate the conclusions beyond the local context beyond the caveat of the case study. 
As such, we have refrained from elaborate statistical analysis that would likely be irrelevant in a case study. 
Instead, we have focused on providing this food for thought for managers.

We look forward to working with the editors and reviewers of \journal~to ensure our manuscript meets the standards of the journal. 

\vspace{-3em} 
\closing{} %eg. Regards
%\cc{another dude}
\end{letter}



\end{document}
