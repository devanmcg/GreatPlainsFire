\documentclass[parskip=half, american]{scrartcl}
\usepackage{booktabs}
\usepackage[margin=2.5cm]{geometry}
\usepackage[dvipsnames]{xcolor}
\usepackage[colorlinks=true,
			citecolor=blue,
			urlcolor=blue,
			linkcolor=blue,
			pdfborder={0 0 0}]
	{hyperref}
	
\newcommand{\AR}[1]
	{\color{PineGreen}AR: #1\color{black} \par }
	
\newcommand{\journal}{\emph{Environmental Management}}


\title{\Large Evaluating an attempt to restore summer fire in the Northern Great Plains }
\subtitle{Response to first review}
\author{ }
\date{}

% \pagenumbering{gobble}
\raggedright 

\begin{document}

\maketitle

\vspace{-2cm} 

\AR{We appreciate the opportunity to revise this submission to \journal, and trust the editor will find these revisions address the important reviewer concerns. 
	Author responses (AR:) to each comment follow inline below, as colored text.\\
\vspace{1em}
Major points of revision include:
\begin{itemize}
\item Addition of statistical analysis for all graphed data and discussed relationships
\item Dropped references to burn ``success'' and emphasis on whether burns were completed as planned or not
\item Expanded discussion of land surface phenology and interactions with non-native cool-season grasses
\end{itemize} }


\section*{Reviewer 1}

I have read the manuscript ``Evaluating an attempt to restore summer fire in the Northern Great Plains'' aimed for publication in Environmental Management. 

The study described four years of successful/unsuccessful prescribed burns on North Dakota grasslands. As a way to restore old traditions of a bimodal burn pattern of anthropogenic burns, greenness (in terms of satellite-retrieved NDVI) burn severity (differenced Normalized Burn Ratio) we studied during sprig and summer season burn series over the four years. The authors conclude that (1) spring burn dNBR decerases with increased NDVI, (2) summer burns are possible (50\% success rate in the study material), (3) summer burns can for some years be achieved with high burn severity and (4) 42 years of historic data indicate that conditions (weather and greenness) suitable for summer burns are more of the norm than an anomaly. Thus, late growing season burning could be included in prescribed fire programs (although not always achievable). 

\AR{We appreciate the reviewer's attention to detail and accurate summary of the manuscript. } 

My general positive thoughts on the manuscript are
1.	I really enjoyed reading it. The language is in many parts well-balanced, the text is short without unnecessary parts, and I found the results interesting.
2.	The study is well limited and the method sound.
3.	The topic is interesting and important.
4.	The study is based on an assessment of plenty field work. 

\AR{ We greatly appreciate this feedback. }

On the other hand, although I praised the short and concise text, as interested in grassland fires, I feel some info on the experiments are missing:
5.	In addition to the NDVI, did you conduct and other measurement on curing degree? Either destructive sampling, visual assessment or levy-rod transects (as detailed in Anderson et al (2011) Int J Wildland Fire 20, 804–814). 

\AR{In short, no. Fuels data for a subset of the spring fires were reported in the cited publication \emph{McGranahan et al 2023 Weather and fuel as modulators of grassland fire behavior in the northern Great Plains. Environmental Management 71:940–949}, but the present manuscript considers a broader suite of fires for which field data are not available.   }

6.	Were there any moisture content measurements of full fuel beds or the dead components during the burns?

\AR{In short, no; see response above. }

7.	Although I appreciate and respect that 48 burning attempts and 40 successful burns constitute a huge workload and plenty of data, the conclusion seems somewhat based on anecdotal evidence. This could however be covered if the authors dwell a bit on the generalisation to these specific two grass species, which are stated to be invasive. This is not clear from someone not familiar with North Dakota grasslands.  

\AR{We agree with the reviewer's assessment. 
This is the main reason that we have submitted as a ``Methodology'' type manuscript, rather than an ``Original research paper'', which the reviewer might not have been aware of. 
Anyway, we recognize the limitations of extrapolating study results and offer the paper more as a case study on feasibility of burning in the summer, and effects of summer fires relative to spring burns. 
We have attempted to highlight this caveat throughout the revision.   }

However, as the topic is important and I feel the results should be published, I recommend acceptance subject to revisions. Below you will find some more detailed comments.

8.	Abstract: ``We found that burn severity declined with fuelbed greenness but was independent of burn season—summer burns could effect as high of severity as spring burns despite having greener fuelbeds''. The sentence makes sense but is a bit difficult at first reading and also when assessing the figures. Severity declines with greenness, but is independent on seasons. Summer burns have a high greenness but equally high severity. Thus, severity does not decline with greenness, or only within spring fires?

\AR{We have rewritten and restructured the first clause to accommodate the seasonal differences described in the second clause and depicted in the figures.  }

9.	Abstract: Next sentence. I would recommend adding a time period to the phenological change (such as ``four decades'' or so), as to not confuse it with changes over a single season. 

\AR{Good point. We've added ``over 42 years''.  }

10.	Methods. Study location: In the introduction you discuss historical burn patterns but the grass species you describe is a non-native species. Could you dwell a bit on this, either here, in the introduction or the discussion. It should not be left uncommented. 

\AR{ The effects of non-native grasses on fire behavior were specifically considered in the Discussion in the original submission, and this discussion remains in the revised version, as well.
Along with this and the other reviewer's suggestion, we have added material to the Discussion about drought susceptibility. }

11.	Methods. Study location: For someone not familiar with the local grasslands you are studying, could you provide us with a short description of these species in for this location? Examples include litter fuel load in spring, annual production, a characteristic season (growth peak, curing, onset of growth), characteristic height of the fuelbed (spring height after snow compression and typical height during the summer burns). 

\AR{ While we appreciate the reviewer's interest, these details go beyond the scope of this paper and frankly beyond the scope of the paper already reporting specific components of the wildland fire environment.  }

12.	Methods. Study location: Photos would be helpful. If you don’t want to fill the manuscript with more figures, there is plenty of room in the supplementary information, which now only contains code. 

\AR{ It is unclear what sort of photos the reviewer would find helpful. 
While we could fill the Supplementary Information with all sorts of cellphone photos from prescribed fire operations, their utility to the manuscript are unclear. }

13.	Methods. Data: Was there no post fire ground assessment of the data? Although dNBR is a good tool for burn severity over large areas did you sample e.g. remaining fuel after burns? This is especially interesting for the summer burns. 

\AR{No ground assessments were conducted.  }

14.	Methods. Data. Line 136: (1000-1700) is this time. I would recommend 10:00 – 17:00 or 10.00 – 17.00 and the time zone this is evaluated in to avoid confusion. 

\AR{Presentation of the burn period has been revised. 
Time zone is irrelevant in this context\textemdash in fact time zones exist such that times can be standardized within the diurnal period.  }

15.	Results: Please describe the aftermath of summer burns. For low severity, did only cured and semi-cured fuel burn while green was not consumed, or was the fuel bed consumed completely for patches were fire could spread but low severity implies a more mosaic post-burn fuel with more green areas and less consumed areas?

\AR{Such qualitative assessments were not made.  }

16.	Discussion: Please discuss why the burn severity is so high during the 

\AR{ Unfortunately the reviewer's comment is abbreviated and remains too vague to infer what was meant. }

17.	Discussion: Lines 199-202: I think you should dwell a bit more on this. Can you assess the magnitude of these changes compared to the variables (greenness and weather) you studied here?

\AR{We appreciate the reviewer's focus on this important topic. 
It is unclear what additional dwelling might be done beyond the specific effects listed here. 
The cited papers, especially Yurkonis et al. 2019, substantiate this list\textemdash indeed, `assess the magnitude' as the reviewer says\textemdash  by using fire behavior models to show that early green-up reduces rate of spread. 
We feel that the point has been sufficiently delivered here based on the literature we are aware of.  }

18.	Results or Discussion: Please discuss also the effect of grazing. Was it uniform given your relatively large burn plots? Is one of the species more attractive to grazing then the other and does that influence the amount of curing?

\AR{We have no data on grazing $\times$ fire environment interactions. 
In general we have no ancedotal evidence that there was any particular effect of cattle being on these pastures\textemdash other than general reductions in potential standing biomass through offtake\textemdash nor do we have any data or observations related to the specific points posited by the reviewer.
It might be worth remembering that in most rangelands, grazing is the norm, not the exception.    }

19.	Figure 1: Summer pattern is described by only eight data points. If the 2017 data with NDVI~0.42 is ignored you’ll retrieve a strong burn severity dependence on NDVI. Could you assess the ground data of that point? Any differences in height, fuel load or pressure from the pastures?

\AR{ It is unclear what the reviewer is referring to\textemdash all summer NDVI values, 2017 \& 2018, are well above 0.42. 
If the reviewer meant to say \emph{dNBR} value above 0.42, we point out that the two 2017 dNBR values that exceed 0.4 represent the lowest and highest values of the NDVI range, so striking values above 0.42 would not improve the linear fit.  
Either way, unfortunately ground-level assessments of past fires is not possible. 
To our knowledge these pastures had been managed consistently prior to and during our study. }

20.	Figure 3. Caption: (Minor comment) the boxplots do not represent historical trends, do they? I would consider them being historical values rather than trends.  

\AR{While we feel that comparing a latter time period to an earlier one does seem to count as a trend, we take the point and have revised to more accurately speak to the comparison of the endpoints.    }

Finally, after this list of potential improvements, I'd like to repeat that I enjoyed reading the manuscript. 

\AR{We very much appreciate the reviewer's consideration, helpful comments, and support for publication. }

\section*{Reviewer 2}

The authors present an analysis describing the effects of fuel and weather conditions on successful or unsuccessful summer burns and place these burn seasons into historical context.  Their objective is to determine whether summer burning is a possible management option in addition to spring burning.  I have several questions about this analysis.
What defines a successful or unsuccessful burn? Is it a certain percentage of a patch burned? Was a patch unburned because the burn boss decided not to that season?  If that’s the case, why didn’t they decide to burn?  

\AR{References to ``success'' have been replaced with specific language about completed vs. not completed burns. }

Is there a significant difference between spring and summer NDVI, cumulative rainfall, dew point, or any of the other variables? How different are the means and variability?  Can burn success be modeled as a function of the variables they describe?  What are the models shown in fig 1b, how well do they fit, and what does the shaded area show?

\AR{Extensive statistical tests have been added for all relationships. }

In addition, I have several line comments

63:  This is an interesting paragraph, but it seems out of place between the previous and next paragraphs.  Either moving it up to line 49 or into the study area part of the methods may fit better.

\AR{After consideration, we disagree with moving the paragraph. 
 	It ought not go before the previous paragraph (L 49 as the reviewer first suggested) because the focus is on the Northern Great Plains specifically, while the previous paragraph discusses the entire Great Plains and includes references to similar grassland systems worldwide. 
Thus moving it up would re-expand the narrowing scope of the Introduction. 
Moving it out of the Introduction entirely is not desirable, because in addition to framing our geographic focus\textemdash indeed, suitable for Methods material\textemdash this paragraph makes the socio-ecological case for this narrower geographic region still being of interest to an international audience because of the extreme discrepency between what is understood to be the pre-colonial fire regime and the one that has emerged from conventional management (and is thus so sought to be reversed by the topic of our study, restoring summer burning). 
This is a core conceptual aspect that merits inclusion in the Introduction as a selling point for the interesting components of our paper.  }

78.  What do you mean by `successful burn'?  

\AR{We appreciate this opportunity to clarify the language in the paper. 
It was our intention that readers would understand our description in the first clause of the sentence\textemdash ``all spring burns were completed as planned''\textemdash to be synonymous with ``success'' in the second clause\textemdash i.e. \emph{not} completed as planned\textemdash but apparently this was not clear. 
We've revised to be more explicit, attempting generally to refrain from the word ``success''.   }

80.  So this objective really seems to hinge on the definition of success.  Are they successful because they were ignited and burned the specified area? Or unsuccessful because they were ignited and didn't burn?  

\AR{The former. We've revised to simply speak of completing or not completing planned burns.  }

90 \& 96.  There are two strategies that involve spring burning – were both strategies reflected in the 32 planned spring burns, or are those just for the spring + summer burning?  

\AR{All spring burns are included in the 32 spring burns.  }

92.  Livestock access is an unreported, possibly important variable.  Why wasn't it included, and if it’s not impactful, a line saying why it’s not is needed.

\AR{All pastures were grazed under exactly the same stocking regime. 
Without variability or comparison, there is little to be said quantitatively about any effects of grazing beyond the fact that it existed, which we have done. No relevant qualitative observations come to mind. It seems a slippery slope to start adding lines about why each thing that was not `impactful' was not considered; we've focused our discussion on elements that are relevant.
Again, as mentioned above, it is the norm for rangelands to be grazed, not the exception.  }

99.  How did the manner of ignition impact success/unsuccessful summer fires?

\AR{In short, it did not. Ignition patterns were selected and adapted to facilitate fire spread. }

136.  1000-1700 hours? 

\AR{ Indeed, revised accordingly. }

146. How sensitive is the analysis to the selection of decade period?  If 1991-2000 were chosen, would the results change?  Are there any anomalous weather years that should be excluded from each decade? How were they summarized?

\AR{While there is a potential for different years slightly changing summary statistics, the nature of central tendency ensures the effects on the overall trends reported here would be minimal. 
We are unaware of `anomalous' weather years and are not sure what that would mean\textemdash it seems beyond the scope of this paper to define a metric for anomalous deviation from some baseline and create selection criteria to determine if a given year should be excluded from a 10-year mean. 
We are simply comparing the mean of a decent chunk of time at the beginning of the available weather period to a comparable chunk of time at the end. 
It seems excessive, especially for a case report from a single weather station, to create a bunch of rules for inclusion/exclusion when there is no apparent need.    }

152.  So a linear models is presented in figure 1b – what are the details of these models?  What are the slopes, r2,  etc?  This sort of information is important for the reader to make their own determination about whether a linear analysis is appropriate.  What about subsampling each fire and use a mixed effects model? 

\AR{ Extensive statistical tests have been added for all relationships. }

156 -166.  These statements all need some sort of statistic or values with them.  What are the typical ranges? What is atypically high?  Are these statistically significant differences by some sort of comparative test? 

\AR{Extensive statistical tests have been added for all relationships.  }

168.  This is something for the discussion rather than the results. 

\AR{This has been removed.  }

177.  Again, if success is due to a burn manager deciding not to burn, then the analysis and its interpretation changes.  If it’s due to a percent unburned or incapable of burning, then this statement is much more impactful. 

\AR{ We appeciate the reviewer's insight into helping us focus on the impactful elements of the story here. 
Indeed, the issue is ``incapable of burning.''
In addition to generally removing references to ``success'', we have added an anecdote that reinforces the barrier presented by environmental conditions in the summer\textemdash attempting to burn but being thwarted by high relative humidity. }

186.  What does `high end' mean? Top 90\%, 95\%?  

\AR{This has been revised with reference to the calculated $z$ values. }

197.  this is an interesting point that I’d like to see expanded on with references.  What have other’s written about differences between expected changes between spring and summer climates?

\AR{This is a great suggestion, thank you. 
We have substantially revised this general section of the Discussion, incorporating both greater specificity of our data with the added statistical analyses, and several references on the topic of ``land surface phenology'', a new concept we found in the literature review prompted by this comment that does well to encapsulate the approach we employ here, and couch our discussion of fire seasonality within a broader ecological literature.  }

334. Is there a sample size for each point/fire? A table with the size/\# pixels for each fire and some additional descriptions (dates, ignition type, etc) in the supplemental would be helpful. What does the shaded region indicate, and how was it calculated? Adding the regression equation as an insert with r2 would help as well.

\AR{No\textemdash data for all burns are summarized to the experimental unit (8 or 16 ha patches). 
Revisions address comments about statistics.   }



\end{document}
