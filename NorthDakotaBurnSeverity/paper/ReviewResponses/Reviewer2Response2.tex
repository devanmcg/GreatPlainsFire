\documentclass[parskip=half, american]{scrartcl}
\usepackage{booktabs}
\usepackage[margin=2.5cm]{geometry}
\usepackage[dvipsnames]{xcolor}
\usepackage[colorlinks=true,
			citecolor=blue,
			urlcolor=blue,
			linkcolor=blue,
			pdfborder={0 0 0}]
	{hyperref}
	
\newcommand{\AR}[1]
	{\color{PineGreen}AR: #1\color{black} \par }
	
\newcommand{\journal}{\emph{Fire}}


\title{\Large Remote sensing in rangeland fire ecology: Comparing imagery to measured fire behavior, and burn severity across prescribed burns and wildfires  }
\subtitle{Response to second review}
\author{ }
\date{}

\pagenumbering{gobble}
\raggedright 

\begin{document}

\maketitle

\vspace{-2cm} 

\AR{Author Responses (AR) appear below in this green text.}

\section*{Reviewer 2}

Thank you for the opportunity to review this manuscript. I have read the authors’ responses, and I would like to reiterate my key concerns in light of their replies.

1. On the authors’ response style
The tone of the author’s reply is rather combative in several places—for instance, dismissing suggestions as “self-contradictory” or claiming not to understand their purpose. This does not foster constructive revision and is unhelpful in moving the manuscript forward.

\AR{While I certainly did not mean to come across as ``combative'' and apologise for any hurt feelings, the reviewer ought not to take offense to having instances in which comments are not helpful too personally.}

2. Specific issues that must be addressed

(1) Introduction

The current version remains too broad. Please sharpen the focus by trimming general background and instead clearly position the study within the relevant literature. A more explicit comparison of key domestic and international work would help clarify the research gap. Most importantly, the specific novelty and necessity of integrating field sensors with satellite dNBR in rangeland fire studies should be stated more directly.

\AR{Although  I disagree with the reviewer, I have removed text placing the study of wildland fire within the broad historical trends of science.
	THe briefer Introduction directly addresses the ``specific novelty and necessity'' of the comparison focused on here. \\
	
	But the reviewer does not at all provide the additional details I requested about what justifies or necessitates an explicit discussion of ``domestic'' vs ``international'' work, which is a distinction I have never seen in any Introduction that does not have as its stated goal such a comparison. \\
	
	To the reviewer's final point, I have made a direct statement about the novelty and necessity of integrating field sensors with satellite data.  }

(2) Methods

dNBR image selection: The description is still vague. Please specify concrete temporal criteria (e.g., ``the last cloud-free image within 30 days before ignition'' and ``the first cloud-free image from the first post-fire growing season'').

\AR{But these statements are simply not true.
	The details the reviewer adds are incorrect\textemdash they are not steps in the process.
	The process is exactly as described\textemdash the description is not vague, the process is simple.}
	

Sampling rationale: The choice of a 20-m buffer (``two Sentinel-2 pixels'') and ~100 sample points remains arbitrary without methodological or literature support. Please justify these choices explicitly—for example, by citing similar buffer distances used in comparable studies or by explaining how the sample size was determined.

\AR{A citation to a publication that used this same methodology has been added.} 

(3) Results
The reported ``positive correlation'' would be more informative if accompanied by significance levels (p-values) alongside the regression coefficients.

\AR{Both the initial submission and the revised draft the reviewer had access to included the P values; only in response to another review were the regression coefficients added to the existing test statistics and P values. }

(4) Conclusions
The conclusions are still presented as a single dense paragraph. Restructuring them—for instance, using subheadings or bullet points—would greatly improve readability and help readers quickly grasp the key takeaways.

\AR{The Conclusions have now been broken into two paragraphs, and a set of bullet points have been added in a ``Key takeaway'' section as well, per the reviewer's request.   }





\end{document}
