\documentclass[parskip=half, american]{scrartcl}
\usepackage{booktabs}
\usepackage[margin=2.5cm]{geometry}
\usepackage[dvipsnames]{xcolor}
\usepackage[colorlinks=true,
			citecolor=blue,
			urlcolor=blue,
			linkcolor=blue,
			pdfborder={0 0 0}]
	{hyperref}
	
\newcommand{\AR}[1]
	{\color{PineGreen}AR: #1\color{black} \par }
	
\newcommand{\journal}{\emph{Fire}}


\title{\Large Remote sensing in rangeland fire ecology: Comparing imagery to measured fire behavior, and burn severity across prescribed burns and wildfires }
\subtitle{Response to first review}
\author{ }
\date{}

\pagenumbering{gobble}
\raggedright 

\begin{document}

\maketitle

\vspace{-2cm} 

\AR{Author Responses (AR) appear below in this green text.}

\section*{Reviewer 3}

Overall

This is a very interesting paper and can contribute to the body of knowledge for rangeland fire science. I would like the authors to consider simply using the word ``severity'' instead of ``behavior'' (as noted below) throughout the paper.  Behavior should be used when writing about active fires, rate of spread, smoke production, etc.

\AR{Indeed, the reviewer is quite correct: Behavior is best limited to the suite of properties used to describe active fires, which is the case in this paper.
Fire behavior refers to observed temperature attributable to combustion and the rate of spread; the limitations of the former are explored in depth.  }

Second, since a relationship between dNBR and heat energy is not expected and pretty much impossible to glean from dNBR, I wonder why this was even considered for this study.  I think the paper would be just as strong, if not stronger, by removing this.  Or perhaps it is used to simply demonstrate that ``it does now work''.

\AR{The reviewer is correct about heat energy not being a component of NBR. 
See below for a complete response to this point. 
As for whether it ``works'' or not, the statistical analysis confirms there is a correlation between temperatures measured during burns and the $\Delta$NBR value observed at that point.  }


Abstract

line 6 a fire severity model does not model behavior but severity. I suggest changing the word behavior to severity. 

\AR{The facts the reviewer relies upon are correct, but they seem to have misread the text. 
I make no mention of any model, much less a ``fire severity model'' specifically. 
I refer first to methods to determine burn severity, and point out that there has been no  attempt to see if values derived from such assessments correlate with measured fire behavior. }

line 14 does not read well, and use OF of those....???

\AR{Thank you for pointing this out. 
   I meant ``uses'' instead of ``use''. }

Introduction 

line 49 I have been working in wildfire science for decades and have not seen ``behavior'' defined to include fire intensity (energy release) or severity.  Fire behavior is typically used to describe how an active fire burns (rate of spread, etc.). Is this needed in the paper?  Why not simply use the term ``severity''?

\AR{I appreciate the reviewer's attention to these definitions. 
	Standard definitions of fire behavior most certainly include fire intensity and energy release; e.g., the Finney et al. text cited at the point referred to above, \emph{Wildland Fire Behaviour}, includes phrases such as ``...how fires behave\textemdash that is, how they spread and release energy (p. 21)''; chapter 3 elaborates on the first-principles aspect of thermodynamics in the definition of fire behavior. 
	The other text cited is more explicit: ``Fire behaviour describes energy release by the combustion of vegetation (p. 22).''
	Nowhere in the sentence defining fire behavior is severity mentioned or included. } 

Methods

This section reads quite well

\AR{I appreciate the feedback.   }

Results

I would not expect dNBR to correlate with temperature.
It is not collected during the fire and does not use any thermal bands in the calculation of this band ratio. 
Is this analysis even necessary and should it be removed from the paper.

\AR{The reviewer is correct, no component of NBR reflects energy release and no NBR data are used from during the period a fire burns. 
The intent here is to determine if there is any relationship between data on fire behavior collected while the fire is occurring, and the differences that are detected as a result of that fire having occurred.
This is in fact the first of the two primary objectives of the paper, so removing it would be counter-productive and would preclude managers and researchers having this essential piece of information.  }

Discussion

line 330 Correlation does not infer causality. You are seeing a secondary or indirect relationship. The higher temperature burns (higher intensity) are most likely due to more fuel being burned more completely resulting in higher severity.  It is not a perfect relationship though as expected.

\AR{The reviewer is correct in each of these points.  }

Conclusions

This is well written and supported by the data.

\AR{I appreciate the reviewer's consideration and feedback.   }



\end{document}
