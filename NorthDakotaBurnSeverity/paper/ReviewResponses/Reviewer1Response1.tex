\documentclass[parskip=half, american]{scrartcl}
\usepackage{booktabs}
\usepackage[margin=2.5cm]{geometry}
\usepackage[dvipsnames]{xcolor}
\usepackage[colorlinks=true,
			citecolor=blue,
			urlcolor=blue,
			linkcolor=blue,
			pdfborder={0 0 0}]
	{hyperref}
	
\newcommand{\AR}[1]
	{\color{PineGreen}AR: #1\color{black} \par }
	
\newcommand{\journal}{\emph{Fire}}


\title{\Large Remote sensing in rangeland fire ecology: Comparing imagery to measured fire behavior, and burn severity across prescribed burns and wildfires  }
\subtitle{Response to first review}
\author{ }
\date{}

\pagenumbering{gobble}
\raggedright 

\begin{document}

\maketitle

\vspace{-2cm} 

\AR{Author Responses (AR) appear below in this green text.}

\section*{Reviewer 1}

This paper focuses on the grassland fire ecology in the northern Great Plains of the United States. By combining $\Delta$NBR remote sensing indicators with on-site fire behavior measurements, it compares the severity of planned burning with that of wildfires, making up for the deficiencies in the field of grassland fire ecology research and demonstrating certain innovation and practical value. The research methods are scientific and reasonable, the data analysis process is clear, and the results are reliable. The conclusion has high reference value for grassland fire monitoring and management.



Advantages:

The research topic is practical. In response to the challenges of fire behavior measurement in grassland ecosystems, a feasible alternative method is proposed, providing a new idea for grassland fire management.

The assessment dimensions are comprehensive, verifying the association between $\Delta$NBR and fire behavior, comparing the differences in fire types by region, and the statistical analysis is rigorous.

The conclusion is clear, highlighting the application value of remote sensing indicators, providing a reference for grassland fire management and post-fire recovery, and indicating the future research direction.


\AR{ I appreciate the support for this manuscript.  }


Shortcomings and Suggestions:

It is suggested that the calculation formula, principle of NBR (Normalized Combustion Index) and its scientific basis in the assessment of fire severity be elaborated in detail in the method section, fully explaining this core indicator.

\AR{I appreciate the reviewer's attention to these details.
The equation for the NBR index and considerable more information about the value behind each band has been added to the Methods section.    }

It is suggested that the limitations of this method be explored in more depth in the discussion section. For instance, the 10× 10-meter resolution of remote sensing data may fail to capture the fine-grained changes in fire behavior in grassland ecosystems, as well as the impact of grassland fuel heterogeneity on the assessment of combustion severity.

\AR{This is a good point.
The Discussion includes a paragraph on pixel-scale averaging of burn severity and the potential loss of fine-grain heterogeneity in fire behavior and fire effects.   }

The current relationship model between remote sensing indices and fire behavior is based on a specific region (the Great Plains of the northern United States), and its applicability under different grassland types (such as tall grassland and desert grassland) and different climatic conditions still needs further verification. It is suggested that the research area be expanded in subsequent studies and verification work be carried out in different ecological zones.

\AR{Thank you, I appreciate this point. 
I agree that it is essential for subsequent studies to expand the research into different ecological zones.   }



\end{document}
